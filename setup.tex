\usepackage[english]{babel}
\usepackage[utf8]{inputenc}
\usepackage[T1]{fontenc}
\usepackage[includefoot, a4paper,left=3cm,right=2.5cm,top=3cm,bottom=3cm, twoside]{geometry}
\usepackage[dvipsnames,table]{xcolor}
\usepackage{stmaryrd}
\usepackage{amssymb}
\usepackage{amsmath, nccmath}
\usepackage{dsfont}
\usepackage{bm}
\usepackage{libertine}
\usepackage[pdftex]{graphicx}
\usepackage{caption}
\usepackage{subcaption}
\usepackage{float}
\usepackage{multirow}
\usepackage{makecell}
\usepackage{siunitx}
\usepackage{apalike}
\usepackage{wrapfig}
\usepackage{tikz}
\usetikzlibrary{backgrounds}
\usetikzlibrary{arrows,shapes}
\usetikzlibrary{tikzmark}
\usetikzlibrary{calc}
\newcommand{\highlight}[2]{\colorbox{#1!17}{$\displaystyle #2$}}
%\newcommand{\highlight}[2]{\colorbox{#1!17}{$#2$}}
\newcommand{\highlightdark}[2]{\colorbox{#1!47}{$\displaystyle #2$}}

% Commands for Highlighting text -- non tikz method
\renewcommand{\highlight}[2]{\colorbox{#1!17}{#2}}
\renewcommand{\highlightdark}[2]{\colorbox{#1!47}{#2}}
\usepackage{enumitem}
\usepackage{epigraph}
%\usetikzlibrary{shadows.blur}
\usepackage{lscape}
\usepackage{mdframed}
\usepackage{titletoc}
\usepackage{tocloft}
%\usepackage{lipsum}
%\usepackage{cite}
\usepackage{booktabs}
\usepackage{arydshln}
\usepackage{fontawesome} 
\usepackage{calc}
\usepackage[newparttoc]{titlesec} 
\usepackage[export]{adjustbox}
\usepackage[bottom]{footmisc}
\usepackage{appendix}
\usepackage{tcolorbox}
\usepackage[onehalfspacing]{setspace}
\usepackage[colorlinks=true,
citecolor=linkColor,linkcolor=linkColor]{hyperref}
\usepackage{fancyhdr}
\usepackage{lipsum}
\usepackage{tcolorbox}
\tcbuselibrary{breakable} 
\usepackage[capitalize,nameinlink]{cleveref}
\crefname{section}{Sec.}{Secs.}
\Crefname{section}{Section}{Sections}
\Crefname{table}{Table}{Tables}
\crefname{table}{Tab.}{Tabs.}
\crefname{property}{Property}{Properties}
\crefname{appproperty}{Property}{Properties}
\crefname{appendix}{App.}{Apps.}
\crefname{chapter}{Chap.}{Chaps.}
\Crefname{chapter}{Chapter}{Chapters}
\crefname{part}{Part}{Parts}
% \creflabelformat{section}{#2\thechapter.#1#3}
% \Crefname{part}{Chapter}{Chapters}

\usepackage{tabto}
\usepackage{pdfpages}

\usepackage{tabularx}
\usepackage{array}

\usepackage{algorithm}
\usepackage{algpseudocode}

\usepackage{bookmark}
\usepackage{rotating}

\usepackage{stmaryrd}
\usepackage[normalem]{ulem}
\useunder{\uline}{\ul}{}

\newcolumntype{P}[1]{>{\centering\arraybackslash}m{#1}}

\newcommand{\midsepremove}{\aboverulesep = 0mm \belowrulesep = 0mm}
\midsepremove
\newcommand{\midsepdefault}{\aboverulesep = 0.605mm \belowrulesep = 0.984mm}
\midsepdefault

\usepackage{ntheorem}
\theoremstyle{break}
\newtheorem{property}{Property}
\newtheorem{appproperty}{Property}
\newtheorem{proposition}{Proposition}
\newtheorem{appproposition}{Proposition}
\theoremstyle{nonumberplain}
\newtheorem{proof}{Proof}

% toc depth parameters
\setcounter{secnumdepth}{2}
\setcounter{tocdepth}{1}


\interfootnotelinepenalty=10000

%%%%%%%%%%%%%%%%%%%%
% Color definition
%%%%%%%%%%%%%%%%%%%%
\definecolor{linkColor}{HTML}{000000}%black
\definecolor{secnum}{RGB}{13,151,225}
\definecolor{ptcbackground}{RGB}{240,240,240}
\definecolor{ptctitle}{RGB}{0,177,235}
\definecolor{THEMECOLOR}{HTML}{750000}

\newcommand{\maj}[1]{{\color{red!40!yellow}{{#1}}}}

\newcommand{\rbf}[1]{\textcolor{red}{\textbf{#1}}}
\newcommand{\bbf}[1]{\textcolor{blue}{\textbf{#1}}}

\newcommand{\rtd}{\textbf{\textcolor{BrickRed}{\faThumbsDown}}}
\newcommand{\gtu}{\textbf{\textcolor{OliveGreen}{\faThumbsUp}}}

\newcommand{\bfsc}[1]{\textsc{\textbf{#1}}}


%%%%%%%%%%%%%%%%%%%%%%%%%%%%%%%%%%%%%%%%%%%%%%%%%%%%%%%%%%%
\renewbibmacro{in:}{%
	\ifentrytype{article}
	{}
	{\bibstring{in}%
		\printunit{\intitlepunct}}}

%%%%%%%%%%%%%%%%%%%%%%%%%%%%%%%%%%%%%%%%%%%%%%%%%%%%%%%%%%%%%
% Format titles of various depth (chapters, sections, subsections)
\titleformat{\paragraph}
{\normalfont\normalsize\bfseries}{\theparagraph}{1em}{}
% \titlespacing*{\paragraph}
% {0pt}{3.25ex plus 1ex minus .2ex}{1.5ex plus .2ex}

%%%%%%%%%%%%%%%%%%%%%%%%%%%%%%%%%%%%%%%%%%%%%%%%%%%%%%%%%%%%%%%%%%%%%%%%%%
\titleformat{\chapter}[display]
{\normalfont\color{THEMECOLOR}}
{\Huge\color{THEMECOLOR}\bfseries\sffamily\textsc\chaptertitlename\hspace*{2mm}%
	\begin{tikzpicture}[baseline={([yshift=-0.7ex]current bounding box.center)}]
		\node[fill=THEMECOLOR,rectangle,text=white, rounded corners=0.8mm] {\thechapter};
	\end{tikzpicture}
}
{1ex}
{\titlerule[1.5pt]\vspace*{1.ex}\Huge\color{black}\sffamily\textsc}
[]

\titleformat{name=\chapter,numberless}[display]
{\normalfont\color{black}}
{}
{1ex}
{\huge\textsc}
[]


\titleformat{\part}[display]
{\huge\bfseries}
{	
	\begin{tikzpicture}[overlay,remember picture,shift=(current page.north west)]
		% \draw[fill=THEMECOLOR] (current page.south west) rectangle (current page.north east);
		% \draw[white, ultra thick] ($current page.south west + (1in,1in)$) rectangle (current page.north east);
		\begin{scope}[x={(current page.north east)},y={(current page.south west)}]
			\draw[fill=THEMECOLOR] (0,0) rectangle (1,1);
			% \draw[white, line width=8pt] ($(0,0)+(1cm,-1cm)$) rectangle ($(1,1)+(-1cm,+1cm)$);
			\draw[white, line width=8pt] ($(1,1)+(-3.7cm,+1cm)$) -- ($(0,1)+(1cm,1cm)$) -- ($(0,0)+(1cm,-1cm)$) -- ($(1,0)+(-1cm,-1cm)$) -- ($(1,1)+(-1cm,+1cm)$) -- ($(1,1)+(-2.5cm,+1cm)$);
			\end{scope}
	\end{tikzpicture}%
	\color{white}\sffamily\textsc\partname\nobreakspace\thepart\\
	\vspace{5mm}\\
	\color{white}\titlerule[3pt]}
{0mm}
{\huge\bfseries\color{white}\sffamily\textsc}
[\vspace{2ex}{\titlerule[3pt]}]

% \titleformat{\part}[display]
% {\huge\bfseries}
% {	
% 	\begin{tikzpicture}[overlay,remember picture,shift=(current page.north west)]
% 		% \draw[fill=THEMECOLOR] (current page.south west) rectangle (current page.north east);
% 		% \draw[white, ultra thick] ($current page.south west + (1in,1in)$) rectangle (current page.north east);
% 		\begin{scope}[x={(current page.north east)},y={(current page.south west)}]
% 			% \draw[fill=THEMECOLOR] (0,0) rectangle (1,1);
% 			% \draw[white, line width=8pt] ($(0,0)+(1cm,-1cm)$) rectangle ($(1,1)+(-1cm,+1cm)$);
% 			\draw[THEMECOLOR, line width=8pt] ($(1,1)+(-4.2cm,+1cm)$) -- ($(0,1)+(1cm,1cm)$) -- ($(0,0)+(1cm,-1cm)$) -- ($(1,0)+(-1cm,-1cm)$) -- ($(1,1)+(-1cm,+1cm)$) -- ($(1,1)+(-3cm,+1cm)$);
% 			\end{scope}
% 	\end{tikzpicture}%
% 	\color{THEMECOLOR}\sffamily\textsc\partname\nobreakspace\thepart\\
% 	\vspace{5mm}\\
% 	\color{THEMECOLOR}\titlerule[3pt]}
% {0mm}
% {\huge\bfseries\color{black}\sffamily\textsc}
% [\color{THEMECOLOR}\vspace{2ex}{\titlerule[3pt]}]

%%%%%%%%%%%%%%%%%%%%%%%%%%
% Title format in toc
%%%%%%%%%%%%%%%%%%%%%%%%%%


\titlecontents{part}[0pc]
{\vspace*{8mm}}
{\color{THEMECOLOR}\large\sffamily\bfseries\filcenter \partname~\thecontentslabel: }
{\large\sffamily\bfseries\filcenter \thecontentslabel}
{}
[\color{THEMECOLOR}\titlerule]

% \titlespacing{\chapter}{0pt}{-50pt}{0pt}
\titlespacing{\section}{0mm}{1.5em}{-0.5\parskip}
\titlespacing{\subsection}{0mm}{0.5em}{-\parskip}
\titlespacing{\subsubsection}{0mm}{0.5em}{-\parskip}
\titlespacing{\paragraph}{0mm}{0.5em}{-\parskip}
\titlecontents{psection}[4.5em]{\addvspace{0.0em}}{\color{THEMECOLOR}\contentslabel{2.0em}}{\hspace*{-3em}}{\color{THEMECOLOR}\titlerule*[0.75em]{.}\contentspage}[\vspace{-5pt}]% define format for section of toc; entries: section, left, above, before with label, before without label, filler and page, after
\titlecontents{psubsection}[6.5em]{}{\color{THEMECOLOR}\contentslabel{2.6em}}{\hspace*{-2.6em}}{\color{THEMECOLOR}\titlerule*[0.75em]{.}\contentspage}[\vspace{-5pt}]% define format for subsection of toc; entries: section, left, above, before with label, before without label, filler and page, after



%%%%%%%%%%%%%%%%%%%%%%%%%%%%%%%%%%%%%%%%%%
% Boxes around chapter number in toc
%%%%%%%%%%%%%%%%%%%%%%%%%%%%%%%%%%%%%%%%%%

% \renewcommand{\cftchappresnum}{%
%     \rlap{
% 		% \color{THEMECOLOR}\rule[-\dp\strutbox]{\cftchapnumwidth}{\dp\strutbox+\ht\strutbox+0.4ex}
% 		\begin{tikzpicture}[baseline={([xshift=-1mm,yshift=-0.7ex]current bounding box.center)}]
% 			\node[fill=THEMECOLOR,rectangle,text=white, rounded corners=0.5mm,minimum width=\cftchapnumwidth-5pt, minimum height=\cftchapnumwidth] {\quad};
% 		\end{tikzpicture}
% 	}%
%     \hbox to \cftchapnumwidth\bgroup\color{white}\hss
% 	}
% \renewcommand{\cftchapaftersnum}{\hss\egroup}
% \renewcommand{\cftchapaftersnumb}{\hskip0.3em\relax}

\let\oldmainmatter\mainmatter
 \renewcommand{\mainmatter}{%
  \addtocontents{toc}{\protect\addvspace{15pt}}%
  \oldmainmatter%
 }

\titlecontents{chapter}[0pc]
{\vspace*{3mm}}
{\sffamily\bfseries\hspace{5mm} \rlap{
		\begin{tikzpicture}[baseline={([xshift=-1mm,yshift=-0.7ex]current bounding box.center)}]
			\node[fill=THEMECOLOR,rectangle,text=white,rounded corners=0.5mm,minimum width=\cftchapnumwidth-5pt, minimum height=\cftchapnumwidth]{\thecontentslabel\quad};
		\end{tikzpicture}
	}%
	\qquad}
{\sffamily\bfseries \thecontentslabel}
{\bfseries\hfill\contentspage}

% \titlecontents{chapter}[0pc]
% {\vspace*{3mm}}
% {\sffamily\bfseries\hspace{5mm} \thecontentslabel--}
% {\sffamily\bfseries \thecontentslabel}
% {\bfseries\hfill\contentspage}

% \setlength{\cftchapindent}{5mm}
\setlength{\cftsecindent}{12mm}


%%%%%%%%%%%%%%%%%%%%%%%%%%%%%%%%%%%%%%%%%%%%%%%%%
% Command for partial toc in chapter  front page
%%%%%%%%%%%%%%%%%%%%%%%%%%%%%%%%%%%%%%%%%%%%%%%%%
\newcommand\PartialToC{%
\startcontents[chapters]%
\hypersetup{linkcolor=THEMECOLOR,linktoc=all}%

\begin{mdframed}[backgroundcolor=ptcbackground,hidealllines=true]
\printcontents[chapters]{p}{1}[2]{
	% Make content title in a box with rounded corners only on the left
	\begin{tcolorbox}[width=\textwidth+3\fboxsep\relax, colframe=THEMECOLOR, colback=THEMECOLOR, arc=3mm, sharp corners=east]
		\color{white}\bfseries\sffamily \large\textsc{Contents}
	\end{tcolorbox}
	\vspace{-10mm}
	% \nopagebreak[4]
	
	% \vskip1pt
	}
\end{mdframed}%
}

\newcommand\PartialToCCh{%
\startcontents[chapters]%
\hypersetup{linkcolor=THEMECOLOR,linktoc=all}%

\begin{mdframed}[backgroundcolor=ptcbackground,hidealllines=true]
	\begin{tcolorbox}[width=\textwidth+3\fboxsep\relax, colframe=THEMECOLOR, colback=THEMECOLOR, arc=3mm, sharp corners=east]
			
		\color{white}\bfseries\sffamily \large\textsc{Contents}
	\end{tcolorbox}
	\vspace{-10mm}
	\begingroup
% \let\clearpage\relax
	\let\pagebreak\relax
	\printcontents[chapters]{p}{1}[2]{}
	\endgroup
	\end{mdframed}%

}


\newcommand{\chapabstract}[1]{
	\vspace{-1cm}
    \begin{quote}
        \singlespacing\small
		\large{\textbf{\textcolor{THEMECOLOR}{\textsc{Abstract}}}}\\
        \textcolor{THEMECOLOR}{\rule{13.5cm}{0.5pt}}
        \small{#1}
        \vskip-4mm
        \rule{13.5cm}{0.5pt}
\end{quote}}

%%%%%%%%%%%%%%%%%%%%%%%%%%%%%%%%%%%%%%%%%%%%%%%%%%%%%%%%%%%%%%%%%%%%%%%%%%%%%
\newtcolorbox{rmk}[1][]{
	breakable,
	%title=Remarque,
	colback=white,
	%colbacktitle=green!20!white,
	%left=1cm,
	left skip=1cm,
	coltitle=black,
	fonttitle=\bfseries,
	bottomrule=0pt,
	toprule=0pt,
	leftrule=4pt,
	rightrule=0pt,
	titlerule=0pt,
	arc=0pt,
	outer arc=0pt,
	colframe=THEMECOLOR   
}	

%%%%%%%%%%%%%%%%% 			New commands			%%%%%%%%%%%%%%%%%%%%%%
\newcommand{\HRule}{\rule{\linewidth}{0.7mm}}
\newcommand{\Hrule}{\rule{\linewidth}{0.3mm}}

\newcommand{\ie}{\emph{i.e.}~}
\newcommand{\eg}{\emph{e.g.}~}
\DeclareMathOperator*{\argmax}{arg\,max}
\DeclareMathOperator*{\argmin}{arg\,min}


\addto\captionsenglish{% Replace "english" with the language you use
  \renewcommand{\contentsname}%
    {\rmfamily\mdseries\textsc{Table of Contents}\\ \color{THEMECOLOR}\rule{\textwidth}{2pt}}%
}

\addto\captionsenglish{% Replace "english" with the language you use
  \renewcommand{\listfigurename}%
    {\rmfamily\mdseries\textsc{List of Figures}\\ \color{THEMECOLOR}\rule{\textwidth}{2pt}}%
}

\addto\captionsenglish{% Replace "english" with the language you use
  \renewcommand{\listtablename}%
    {\rmfamily\mdseries\textsc{List of Tables}\\ \color{THEMECOLOR}\rule{\textwidth}{2pt}}%
}

\setlength\parindent{0pt}
\setlength{\abovedisplayskip}{3pt}
\setlength{\belowdisplayskip}{3pt}

% \renewcommand{\chaptermark}[1]{\markboth{\textsc{#1}}{}}
% \renewcommand{\chaptermark}[1]{\markright{#1}}


\usepackage{etoolbox}
% \makeatletter
% \patchcmd{\chapter}{\thispagestyle{plain}}{\thispagestyle{partplain}}{}{}
% \makeatother

\assignpagestyle{\part}{partplain}


% \renewcommand{\thesection}{\arabic{section}}
\setcounter{secnumdepth}{4}
\setcounter{chapter}{1}

\providecommand{\keywords}[1]{\textbf{\textit{Keywords---}} #1}

\providecommand{\keywordsfr}[1]{\textbf{\textit{Mots-Clé---}} #1}


\usepackage[section=section]{glossaries}
% \usepackage{glossary-longbooktabs}
% \setglossarystyle{alttree}
\setlength{\glsdescwidth}{12cm}

% \renewcommand*{\glsclearpage}{\clearpage}

\newglossary{symbols}{syo}{syi}{List of Symbols}    % custom glossary, type=mysmys
\makenoidxglossaries

%%%%%%%%%%%%%%%%%%%%%%%%%%%%%%%%%%%%%%%%%%%%%%%
%                   Symbols                   %
%%%%%%%%%%%%%%%%%%%%%%%%%%%%%%%%%%%%%%%%%%%%%%%
\glsnoexpandfields

\newglossaryentry{a_symbol}{
    type=symbols,
    sort={a_symbol},
    name={$a$},
    description={A symbol}}





%%%%%%%%%%%%%%%%%%%%%%%%%%%%%%%%%%%%%%%%%%%%%%%
%                   Acronyms                  %
%%%%%%%%%%%%%%%%%%%%%%%%%%%%%%%%%%%%%%%%%%%%%%%
\newglossaryentry{fsod}
{
    name=FSOD,
    description={Few-Shot Object Detection}
}


\glsaddall
\glsfindwidesttoplevelname